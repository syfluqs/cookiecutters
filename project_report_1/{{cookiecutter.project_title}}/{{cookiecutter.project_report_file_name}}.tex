
\documentclass[twoside]{report}
\usepackage{graphicx}
\usepackage{wrapfig}
\usepackage{float}
\usepackage{geometry}
\geometry{a4paper, top=1in, bottom=1.15in, left=1.25in, right=1.25in, includeheadfoot}
\usepackage[bottom]{footmisc}
\usepackage{tabularx}
\usepackage{amsmath,amsfonts,amsthm,amssymb}
\usepackage{mathtools}
\usepackage{enumitem}
\usepackage{siunitx}
\usepackage{luacode}
\usepackage[cache=false,outputdir=dist]{minted}
\usepackage{fontspec}
\usepackage{multirow}
\usepackage{pgfplots}
\pgfplotsset{compat = newest}
\usepackage{xfrac}
\usepackage{fancyvrb}
\usepackage[justification=centering]{caption}
\usepackage{lipsum}
\usepackage{xurl}
\usepackage{caption}
\usepackage{subcaption}
\usepackage{lettrine}
\usepackage{longtable}
\usepackage{rotating}
\usepackage{anyfontsize}
% change base font size
\usepackage[fontsize=13pt]{scrextend}

\usepackage{fancyvrb}

\usepackage[style=numeric]{biblatex}
\addbibresource{bibliography.bib}

% - to adjust margins in the table of contents
% \usepackage[titles]{tocloft}
% \cftsetindents{figure}{0em}{3em}
% \cftsetindents{table}{0em}{3.5em}

\newcommand{\lp}[1]{\directlua{tex.print({#1})}}

\begin{luacode}
    project_title = "{{cookiecutter.project_title}}"
    project_subtitle = "{{cookiecutter.project_subtitle}}"
    pdf_title = "{{cookiecutter.subject_code}} Project Report - {{cookiecutter.project_title}}"
    author = "{{cookiecutter.author}}"
    roll_number = "{{cookiecutter.roll_number}}"
    subject_code = "{{cookiecutter.subject_code}}"
    college_name = "college name"
    dept_name = "dept name"

    first_page_heading = "Project Report"
    abstract_title = "Abstract"
    acknowledgements_title = "Acknowledgements"
    list_of_figures_title = "List of Figures"
    list_of_tables_title = "List of Tables"
    list_of_abbreviations_title = "List of Abbreviations"
    references_title = "References"
\end{luacode}

\usepackage[
    pdfusetitle,
    pdfauthor={\lp{author}},
    pdftitle={\lp{pdf_title}},
    % pdfsubject={The Subject},
    % pdfkeywords={Some Keywords},
    % pdfproducer={Latex},
    pdfcreator={LateX},
    colorlinks=true,
    linkcolor=black!30!blue,
    citecolor=red
]{hyperref}

\defaultfontfeatures{Scale=MatchLowercase}
% \setmainfont{Adobe Garamond Pro}
\setsansfont{Open Sans}
\setmonofont{Menlo}
\newfontfamily\definitionfont{Droid Serif}

\usepackage{footnote}
\usepackage[footnote,acronym,toc]{glossaries}
% \makeglossaries
\glsdisablehyper

\usepackage{fancyhdr}

\fancypagestyle{mypagestyle}{
    \fancyhead[L,R]{}										% Clear page headers
    \fancyhead[LE]{\textsl{\leftmark}}
    \fancyhead[RO]{\textsl{\rightmark}}
    \fancyhead[C]{}
    \fancyfoot{}                                            % clear all footer fields
    \fancyfoot[LE,RO]{\thepage}
    \fancyfoot[LO,RE]{\lp{project_title}}
}
\fancypagestyle{toc}{
    \fancyhf{}%
    \fancyfoot[C]{\thepage}
}

\usepackage[nottoc,notlot,notlof]{tocbibind}

\renewcommand{\headrulewidth}{0pt}
\renewcommand{\footrulewidth}{0.4pt}
\renewcommand{\listfigurename}{\lp{list_of_figures_title}}
\renewcommand{\listtablename}{\lp{list_of_tables_title}}

\usepackage{sectsty}
\allsectionsfont{\centering \normalfont\scshape}
\numberwithin{equation}{section}		% Equationnumbering: section.eq#
\numberwithin{figure}{section}			% Figurenumbering: section.fig#
\numberwithin{table}{section}

% change title formats
\usepackage{titlesec}
\titleformat{\chapter}[block]{\normalfont\huge\bfseries}{\thechapter.}{1em}{\huge}
\titleformat{\section}{\normalfont\scshape\Large}{\thesection}{1em}{\Large}
\titlespacing*{\chapter}{0pt}{0pt}{20pt}

\setcounter{MaxMatrixCols}{15}

\newcommand{\SIsci}[2]{\SI[round-precision=3,round-mode=figures,scientific-notation=true]{#1}{#2}}
\newcommand{\SIrnd}[2]{\SI[round-precision=3,round-mode=places]{#1}{#2}}
\newcommand{\numrnd}[1]{\num[round-precision=3,round-mode=places,scientific-notation=fixed,fixed-exponent=0]{#1}}
\newcommand{\numsci}[1]{\num[round-precision=3,round-mode=figures,scientific-notation=true]{#1}}
\newcommand{\aligncell}[2][c]{\begin{tabular}[#1]{@{}c@{}}#2\end{tabular}}

% #1 = width
% #2 = image path
% #3 = caption
% #4 = reference label
\newcommand{\nonfloatingimage}[4]{
    \begin{minipage}{#1}
        \centering
        \captionsetup{type=figure}
        \includegraphics[width=0.95\textwidth,keepaspectratio]{#2}
        \captionof{figure}{#3} \label{#4}
    \end{minipage}
}

% minted shorthands, #1 = language
\newcommand{\sourcecodefile}[2]{\inputminted[linenos,breaklines]{#1}{#2}}
\newenvironment{sourcecode}[1]
    {\VerbatimEnvironment\begin{minted}[linenos,breaklines,autogobble]{#1}}
    {\end{minted}}
% #1 = label of the box
\newenvironment{sourcecodeoutput}[1]
    {\VerbatimEnvironment\begin{minted}[breaklines,autogobble,frame=lines,
        bgcolor=white!90!lightgray,label=#1,labelposition=topline,
        fontsize=\footnotesize]{text}}
    {\end{minted}}

\graphicspath{ {./assets/} }

\begin{document}
    \pagenumbering{roman}
    \begin{titlepage}
        \centering
        
        \LARGE{\textsc{\lp{first_page_heading}}}
        
        \vspace{\stretch{1}}
        
        \Huge{\textbf{\lp{project_title}}}\\
        \rule{3in}{0.4pt} \\
        \Large{\lp{project_subtitle}}
        
        \vspace{\stretch{1}}

        \begin{center}
            \includegraphics[keepaspectratio,scale=0.5]{front_page_logo.png}
        \end{center}

        \vspace{\stretch{3}}
        
        \Large{
            \textbf{\lp{author}}\\
            \lp{roll_number} \\ 
            \lp{college_name} \\
            \lp{dept_name} \\
        }
        
        \vspace*{\stretch{1}}
        
    \end{titlepage}

    \thispagestyle{plain}

    \newcounter{abstractpage}
    \setcounter{abstractpage}{\value{page}}
    \phantomsection
    \begin{abstract}
        \thispagestyle{plain}
        \setcounter{page}{\value{abstractpage}}
        \addcontentsline{toc}{chapter}{\lp{abstract_title}}
        
        \lipsum[1-1]
        
        \setcounter{abstractpage}{\value{page}}
    \end{abstract}
    
    \renewcommand{\abstractname}{\lp{acknowledgements_title}}
    \clearpage
    \phantomsection
    \begin{abstract}
        \thispagestyle{plain}
        \setcounter{page}{\numexpr\value{abstractpage}+1\relax}
        \addcontentsline{toc}{chapter}{\lp{acknowledgements_title}}

        \lipsum[2-10]

        \setcounter{abstractpage}{\value{page}}
    \end{abstract}

    \setcounter{page}{\value{abstractpage}}
    \stepcounter{page}

    \pagestyle{toc}
    \tableofcontents
    \clearpage
    
    \phantomsection
    \addcontentsline{toc}{chapter}{\listfigurename}
    \listoffigures
    \clearpage
    
    \phantomsection
    \addcontentsline{toc}{chapter}{\listtablename}
    \listoftables
    \clearpage

    \printglossary[type=\acronymtype, title=\lp{list_of_abbreviations_title}, toctitle=\lp{list_of_abbreviations_title}]
    \clearpage

    \pagestyle{mypagestyle}
    \pagenumbering{arabic}

    % Main text starts here

    \chapter{Introduction}

    \section{First Sample Section}
    \lettrine[lines=2]{L}{ettrine} example \lipsum[2-3]

    \newacronym{lcm}{LCM}{Least Common Multiple}
    \newacronym{gcd}{GCD}{Greatest Common Divisor}

    \begin{figure}
        \centering
        \begin{subfigure}[b]{0.45\textwidth}
            \centering
            \includegraphics[width=0.8\textwidth,totalheight=\textheight,keepaspectratio]{front_page_logo.png}
            \caption{Image A}
        \end{subfigure}
        \hfill
        \begin{subfigure}[b]{0.45\textwidth}
            \centering
            \includegraphics[width=0.8\textwidth,totalheight=\textheight,keepaspectratio]{front_page_logo.png}
            \caption{Image B}
        \end{subfigure}
        \caption{Sample Images}
    \end{figure}

    \begin{minipage}{0.95\textwidth}
        \centering
        \captionsetup{type=figure}
        \includegraphics[width=0.8\textwidth,totalheight=\textheight,keepaspectratio]{front_page_logo.png}
        \captionof{figure}{Another Image}
    \end{minipage}

    \section{Second Sample Section}
    \lipsum[1-3]

    \begin{table}
        \begin{center}
            \begin{tabular}{|l|l|l|l|}\hline
                \multirow{10}{*}{numeric literals} & \multirow{5}{*}{integers} & in decimal & \verb|8743| \\ \cline{3-4}
                & & \multirow{2}{*}{in octal} & \verb|0o7464| \\ \cline{4-4}
                & & & \verb|0O103| \\ \cline{3-4}
                & & \multirow{2}{*}{in hexadecimal} & \verb|0x5A0FF| \\ \cline{4-4}
                & & & \verb|0xE0F2| \\ \cline{2-4}
                & \multirow{5}{*}{fractionals} & \multirow{5}{*}{in decimal} & \verb|140.58| \\ \cline{4-4}
                & & & \verb|8.04e7| \\ \cline{4-4}
                & & & \verb|0.347E+12| \\ \cline{4-4}
                & & & \verb|5.47E-12| \\ \cline{4-4}
                & & & \verb|47e22| \\ \cline{1-4}
                \multicolumn{3}{|l|}{\multirow{3}{*}{char literals}} & \verb|'H'| \\ \cline{4-4}
                \multicolumn{3}{|l|}{} & \verb|'\n'| \\ \cline{4-4}          %% here
                \multicolumn{3}{|l|}{} & \verb|'\x65'| \\ \cline{1-4}        %% here
                \multicolumn{3}{|l|}{\multirow{2}{*}{string literals}} & \verb|"bom dia"| \\ \cline{4-4}
                \multicolumn{3}{|l|}{} & \verb|"ouro preto\nmg"| \\ \cline{1-4}          %% here
            \end{tabular}
            \caption{Sample Table}
        \end{center}
    \end{table}
    
    \section{Third Sample Section}
    \newacronym[]{hcf}{HCF}{Highest Common Factor}
    Given a set of numbers, there are elementary methods to compute 
    its \acrlong{lcm}, which is abbreviated \gls{lcm}. Also look 
    \acrfull{gcd} \footnote{Also called \acrfull{hcf} \cite{einstein}}.
    \lipsum[4-5]

    \clearpage
    \printbibliography[title=\lp{references_title},heading=bibintoc]
\end{document}

